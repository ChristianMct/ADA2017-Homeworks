In 1943, the soviets won the Battle of Stalingrad against the nazis, while loosing 800,000 soldiers, twice the number of casualties of the german. In spite of these heavy losses, this soviet victory is considered as a major strategic turning point that lead to the soviet victory against the nazis' invasion. Human history and the history of battles more precisely contains numerous example such as this one that show the importance and complexity of battles. In this work, we make a first step into the understanding of the history battles by studying relationship and patterns between multiple battle features such as the relation between the number of casualties and the result of a battle. We observe how the battles evolve through centuries until today by studying their duration or their results. In order to do these observation, we first describe how we collected the data about more than 7,000 battles from wikipedia and how we parsed and processed these data. We discuss the outcome of these data collection and begin our study by looking how the battles are spread in time and locations. Then, we interest ourselves in the battles evolution by observing their duration, the number of casualties and how indecisive the battles are among the years. We infer the battle's features that have the most important impact on the victory or not of a combatant and show that the most important feature is the number of casualties that a combatant suffers. Afterwards, we rank the countries by the number of years during which they were engaged in a battle and noticed that from 20 centuries, France is the country that fought the most. We also remark that since the independence of the United States in 1776, U.S.A. are the country that is the more involved in battles with an impressive percentage of his history spent in battles.