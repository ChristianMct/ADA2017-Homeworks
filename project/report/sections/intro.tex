In 1943, the soviets won the Battle of Stalingrad against the nazis while loosing 800,000 soldiers, twice the number of german casualties. In spite of these heavy losses, this soviet victory is considered as a major strategic turning point that lead to the soviet victory against the nazis' invasion. Human history contains numerous examples such as this one that show the importance and complexity of battles.

While battle-related casualties are far from being representative of all the damages caused by war, they have the particularity of being the result of a strategic and tactical planning, and are therefore worth analyzing. In this work, we make a first step towards understanding the evolution of the way powers wage battle by studying the trends and relationships between multiple battle-related features such as the initial strengths, the number of casualties or the result type of a battle. 

In order to support these observations, we first describe how we collected and processed data on more than 7,000 battles from the English Wikipedia database dump\footnote{https://meta.wikimedia.org/wiki/Data\_dumps} and provide a short spatial and temporal descriptive analysis of the resulting dataset. Then, we report on our findings by observing trends in duration, number of casualties and how indecisive the battles tends to become over the last thousand years. We isolate the adversary losses as a critical feature determining probability of victory and show that this is interestingly not the case for longer-term victories. Lastly, we compute cumulative time spent in battle for each of our extracted belligerents and provide a ranking of the most warlike ones.  
